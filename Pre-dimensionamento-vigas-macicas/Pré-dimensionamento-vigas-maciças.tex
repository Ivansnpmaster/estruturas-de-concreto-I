Para o pré-dimensionamento da altura de vigas, deve-se observar o número de apoios na qual ela está sujeita. Para obtenção dessa altura, deve-se utilizar as seguintes equações:

\begin{itemize}
	\item Para \textbf{vigas contínuas} com mais de dois apoios, subdivide-se a viga em vigas menores, portanto:

		$$h=c  \cdot \frac{L}{10}$$
		Onde $h$  é a altura da viga, $L$ é o comprimento do trecho e $c$ é 0,75 para vigas nas extremidades e 0,7 para as demais. Adota-se a maior altura encontrada para toda a seção transversal.

	\item Para \textbf{vigas em balanço}, temos:

		$$h=\frac{L}{5}$$
		Onde $h$ é a altura da viga e $L$ seu comprimento.

	\item Para \textbf{vigas biapoiadas}, temos:
		
		$$h=\frac{L}{10}$$
		Onde $h$ é a altura da viga e $L$ seu comprimento.
\end{itemize}

Recomenda-se \textbf{não pré-dimensionar} vigas com menos de 25 $cm$ de altura.

Deve-se atentar, entretanto, que para lajes onde há a necessidade da passagem de tubulações, é necessário ajustar a altura da viga. Por exemplo, o teto de banheiros precisa de tubulações, deve-se considerar a espessura da laje do banheiro e a espessura do local onde ficará a tubulação. Para que as vigas não fiquem aparentes, além do usual método de adotar \textbf{alturas de 5 em 5 $cm$}  para facilitar a montagem da forma, deve-se \textbf{cobrir toda essa espessura onde há a tubulação + laje}. Isso é muito importante.

Adota-se $bw$ $\geqslant$ 14 $cm$ , podendo ser $\geqslant$ 12 $cm$ em casos especiais.