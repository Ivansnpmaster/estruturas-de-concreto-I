Em estruturas de edifícios, os pilares são elementos verticais que tem a função primária de transmitir as \textbf{ações verticais} gravitacionais e de serviço e as \textbf{orizontais (vento)} às fundações, além de conferirem \textbf{estabilidade global} ao edifício. Os pilares usuais dos edifícios apresentam um comportamento de flexo-compressão, sendo as forças normais preponderantes.
Em edifícios de concreto armado, as seções dos pilares são geralmente \textbf{retangulares}.

% inserir imagem

Pilares de seção \textbf{quadrada} ou \textbf{circular} também podem ser considerados em projetos estruturais de edifícios.
Em virtude do tipo de material (concreto) e da solicitação preponderantemente de força de compressão, os pilares apresentam \textbf{rupturas frágeis}. A \textbf{ruína} de uma seção transversal de \textbf{um único pilar} pode ocasionar o \textbf{colapso} progressivo dos demais pavimentos.

As \textbf{disposições} dos pilares na planta de forma de um edifício são importantes, pois, junto com as vigas, formam \textbf{pórticos} que proporcionam \textbf{rigidez} e \textbf{estabilidade global} ao edifício.

Os pilares são peças estruturais que precisam ser projetadas \textbf{cuidadosamente} em termos de resistência, estabilidade e durabilidade, sempre respeitando as diretrizes e recomendações das \textbf{normas técnicas}.

O dimensionamento dos pilares é feito em função dos esforços externos solicitantes de cálculo, que compreendem as forças normais $(N_d)$ e os momentos fletores $(M_{dx}$ e $M_{dy})$.