O material é linear quando obedece à Lei de Hooke, ou seja, quando a tensão é proporcional à deformação $(\sigma=E\cdot \epsilon)$. Considerando-se uma estrutura de concreto armado, a não-linearidade física resulta da resposta não-linear do \textbf{aço} e do \textbf{concreto}.

Além do comportamento não-linear dos materiais, existe um outro fator que é preponderante na análise de edifícios: a \textbf{fissuração}. Por causa da baixa resistência do concreto à tração, é comum o surgimento de fissuras à medida que o carregamento é aplicado à estrutura.

A NBR 6118 - item 15.3: "Princípios básicos de cálculo" é bem clara: a não-linearidade física, presente nas estruturas de concreto armado, deve ser obrigatoriamente considerada.