As estruturas de concreto armado devem ser projetadas, construídas e utilizadas de modo que, sob condições ambientais previstas e respeitadas as condições de manutenção preventiva especificadas no projeto, conservem sua \textbf{segurança}, \textbf{estabilidade} e \textbf{aparência aceitável}, sem exigir medidas extras de manutenção e reparo.

Há duas formas de se analisar estruturalmente uma edificação:

\begin{itemize}
	\item Análise linear;
	\item Análise não-linear.
\end{itemize}

Se fosse feita uma análise puramente linear, o \textbf{deslocamento} resultante seria \textbf{proporcional} ao acréscimo de carga.

A resposta da estrutura em termos de deslocamentos teria um comportamento \textbf{linear} à medida que o carregamento fosse aplicado.

Por outro lado, se fosse efetuada uma análise não-linear, o deslocamento resultante \textbf{não seria proporcional} ao acréscimo de carga. E mais, provavelmente seria \textbf{maior} que o encontrado na análise linear.

Pode-se dizer que uma \textbf{análise não-linear} é um cálculo no qual a resposta da estrutura, seja em deslocamentos, esforços ou tensões, possui um comportamento \textbf{desproporcional} à medida que um carregamento é aplicado.

Os fatores que tornam as análises não-lineares importantes no projeto estrutural de edifícios de concreto armado são:

\begin{itemize}
	\item O concreto armado é um material que possui um comportamento \textbf{essencialmente} não-linear;
	\item Pelas análises não-lineares, é possível simular o comportamento de um edifício de concreto armado de forma muito mais \textbf{realista};
	\item Os elementos estruturais estão cada vez mais \textbf{esbeltos}, de tal forma que as \textbf{não-linearidades}, em muitos casos, passam a ser \textbf{preponderantes}.
\end{itemize}

Dois fatores que geram o comportamento não-linear de uma estrutura à medida que o carregamento é aplicado:

\begin{itemize}
	\item \textbf{Não-linearidade física}: Alteração das \textbf{propriedades} dos materiais que compõem a estrutura;
	\item \textbf{Não-linearidade geométrica}: Alteração da geometria da estrutura.
\end{itemize}