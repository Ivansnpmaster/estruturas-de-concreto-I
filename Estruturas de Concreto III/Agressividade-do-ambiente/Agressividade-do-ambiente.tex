Está relacionada às \textbf{ações físicas} e \textbf{químicas} que atuam sobre as estruturas de concreto, independentemente das \textbf{ações mecânicas}, das variações térmicas, da retração e outras previstas no dimensionamento das estruturas.

Nos projetos das estruturas, a agressividade ambiental deve ser classificada de acordo com a Tabela 6.1 da ABNT NBR 6118 e pode ser avaliada segundo as condições de exposição da estrutura ou de suas partes. Conhecendo o ambiente em que a estrutura será construída, o projetista estrutural pode considerar uma condição de agressividade maior que a tabela.

% inserir tabela

Conforme a NBR 6118 - item 7.4: A durabilidade das estruturas é \textbf{altamente dependente} das características do concreto e da \textbf{espessura} e \textbf{qualidade} do concreto de cobrimento da armadura.

\textbf{Ensaios comprobatórios} de desempenho da durabilidade da estrutura frente ao tipo e classe de agressividade prevista em projeto devem estabelecer os parâmetros mínimos a serem atendidos. Na falta destes e devido à existência de uma \textbf{forte correspondência} entre a \textbf{relação água/cimento} e a \textbf{resistência do concreto} e sua \textbf{durabilidade}, permite-se que sejam adotados os requisitos mínimos da tabela abaixo:

\begin{table}[H]
\centering
\caption{Tabela 7.1 da NBR 6118.}
\begin{tabular}{|c|c|c|c|c|}
\hline
\multirow{2}{*}{Concreto} & \multicolumn{4}{c|}{Classe de Agressividade Ambiental (CAA)}                      \\ \cline{2-5} 
                          		& I                  & II                 & III                & IV                 \\ \hline
Relação a/c               & $\leqslant$ 0,65   & $\leqslant$ 0,6    & $\leqslant$ 0,55   & $\leqslant$ 0,45   \\ \hline
Classe de concreto        & $\geqslant$ C20 & $\geqslant$ C25 & $\geqslant$ C30 & $\geqslant$ C40 \\ \hline
\end{tabular}
\end{table}