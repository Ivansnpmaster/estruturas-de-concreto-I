A \textbf{estabilidade global} de uma estrutura se dá quando \textbf{menores} forem os efeitos de 2ª ordem. Para criar condições de cálculo, as estruturas são definidas de nós fixos ou nós móveis.

\begin{itemize}
	\item \textbf{Estruturas de nós fixos}: Na verdade não são fixos, mas deslocáveis, mas os deslocamentos horizontais são muito pequenos e, por consequência, os \textbf{efeitos globais de 2ª ordem} são \textbf{desprezíveis} (<10\%). Nessas estruturas, basta considerar os \textbf{efeitos locais de 2ª ordem};

	\item \textbf{Estruturas de nós móveis}: Aquelas em que os deslocamentos horizontais \textbf{não são pequenos}, exigindo cálculo dos \textbf{efeito globais de 2ª ordem}.
\end{itemize}

Assim:

\begin{itemize}
	\item \textbf{Nós fixos}: \textbf{Não há} necessidade de se calcular efeitos globais de 2ª ordem;
	\item \textbf{Nós móveis}: \textbf{Há} necessidade de se calcular.
\end{itemize}