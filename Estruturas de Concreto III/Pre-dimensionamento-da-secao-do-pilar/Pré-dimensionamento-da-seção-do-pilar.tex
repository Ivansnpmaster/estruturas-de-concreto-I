Para definição da seção de cada tipo de pilar, utiliza-se o \textbf{pilar do térreo}, que é o que recebe maior carga entre todos os pilares e mantém-se essa seção até o último andar.

Sabendo-se as cargas acima do pilar (variável e permanente), $N_d$ será o valor da reação de apoio com as majorações necessárias. Se as cargas forem pré-dimensionadas, é possível também pré-dimensionar a seção dos pilares em função do tipo de pilar e para aço CA-50.

\begin{itemize}
	\item \textbf{Pilar intermediário}:
		\begin{equation}A_c=\frac{N_d}{0,5\cdot f_{ck}+0,4}\end{equation}
	\item \textbf{Pilar de extremidade} e \textbf{pilar de canto}:
		\begin{equation}A_c=\frac{1,5\cdot N_d}{0,5\cdot f_{ck}+0,4}\end{equation}
\end{itemize}

Onde $A_c$ é a área da seção transversal do pilar, $N_d$ é a força normal de cálculo e $f_{ck}$ é a resistência característica do concreto à compressão.

Lembrando: Sem ter a seção do pilar ainda definida, $N_d$ é calculada apenas com a majoração do concreto:
\begin{equation}N_d=\gamma_f\cdot N_k=1,4\cdot N_k\end{equation}

Sabendo-se a área de concreto, devemos consultar a NBR 6118 para saber a área mínima a ser utilizada e também a medida mínima da menor dimensão do pilar.

Item 13.2.3 - A seção transversal de pilares e pilar-paredes maciços, qualquer que seja sua forma, não pode apresentar dimensão menor que 19 $cm$. Em casos especiais, permite-se considerações de dimensões entre 19 e 14 $cm$, desde que se multipliquem os esforços solicitantes de cálculo a serem considerados no dimensionamento por um coeficiente adicional $\gamma_n$, de acordo com o indicado na tabela 13.1 e seção 11 da norma. Em qualquer caso, \textbf{não se permite} pilar com seção transversal de área inferior a 360 ${cm}^2$.

*inserir tabela

A maior dimensão da seção do pilar deve ser sempre em \textbf{múltiplos de 5 $cm$}.

Sabendo-se o valor de $\gamma_n$, pode-se calcular o valor de $N_d$:
\begin{equation}N_d=\gamma_n\cdot\gamma_f\cdot N_k\end{equation}

Exercício: Pré-dimensionar a seção de um pilar intermediário, sabendo-se que: Aço CA-50; Concreto C25 ($f_{ck}=25\;MPa=250\;kgf/{cm}^2$); $\gamma_f=1,4$; $q=14051,52\;kgf$ e $g=144529,92\;kgf$.

Calcular $b=16\;cm$ e depois $b=19\;cm$. Lembre-se que não queremos pilar-parede (onde $h=5\cdot b$).

Para o pilar com $b=16\;cm$, tem-se:
$$A_c=\frac{N_d}{0,5\cdot f_{ck}+0,4}=\frac{1,4\cdot1,15\cdot(14051,52+144529,92)}{0,5\cdot250+0,4}\approx2036,0137\;{cm}^2$$

Com essa área, checa-se se não é um pilar parede (limite de $5\cdot b=5\cdot16\;cm=80\;cm$):
$$h=\frac{A_c}{b}=\frac{2036,0137\;{cm}^2}{16\;cm}\approx127,2508\;cm\approx130\;cm$$

O valor de $h$ ultrapassou o limite e, dessa forma, o pilar é considerado pilar-parede.

Para o pilar com $b=19\;cm$, tem-se:
$$A_c=\frac{N_d}{0,5\cdot f_{ck}+0,4}=\frac{1,4\cdot(14051,52+144529,92)}{0,5\cdot250+0,4}\approx1770,4466\;{cm}^2$$
$$h=\frac{A_c}{b}=\frac{1770,4466\;{cm}^2}{19\;cm}\approx93,1814\;cm\approx95\;cm$$

O valor de $h$ está dentro do limite ($5\cdot b=5\cdot19\;cm=95\;cm$) e não é considerado pilar-parede.

Exercício: Admitindo-se um pilar intermediário com uma carga axial de $40000\;kgf$; aço CA-50; concreto C25 e $\gamma_f=1,4$ e sabendo-se que o lado maior da seção do pilar é \textbf{duas vezes} o lado menor, qual o valor unitário arredondado para cada lado da seção desse pilar?
$$A_{c1}=\frac{N_d}{0,5\cdot f_{ck}+0,4}=\frac{1,4\cdot40000}{0,5\cdot250+0,4}\approx446,5709\;{cm}^2$$
$$A_{c1}=b\cdot(2\cdot b)$$
$$b=\sqrt{\frac{446,5709\;{cm}^2}{2}}\approx14,9427\;cm\approx15\;cm$$

A área de concreto foi pré-dimensionada sem o $\gamma_n$ (que depende do menor lado do pilar). No caso de $15\;cm$, seria de $\gamma_n=1,20$. É necessário recalcular o valor de área de concreto com esse coeficiente.
$$A_{c2}=\frac{N_d}{0,5\cdot f_{ck}+0,4}=\frac{1,4\cdot1,20\cdot40000}{0,5\cdot250+0,4}\approx535,8851\;{cm}^2$$
$$b=\sqrt{\frac{535,8851\;{cm}^2}{2}}\approx16,3689\;cm\approx17\;cm$$

Por mais que o lado de $17\;cm$ tenha outro $\gamma_n$, ele é menor que o $\gamma_n$ de $15\;cm$ e assim sendo, não é necessário pré-dimensionar a área de concreto novamente. O pilar terá um $b=17\;cm$ e $h=34\;cm\approx35\;cm$ (arredondando de 5 em 5 $cm$ para facilitar a execução).