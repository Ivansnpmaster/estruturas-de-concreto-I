A deflexão lateral produzida por uma carga sobre um pilar compõe o processo conhecido como \textbf{flambagem por flexão}. Ela ocorrerá sempre em torno do eixo de menor momento de inércia de sua seção transversal, pois gerará o maior índice de esbeltez.

Para calcular o índice de esbeltez é necessário o comprimento de flambagem, que é dado de acordo com o tipo de vinculação na base e no topo do pilar.

*Inserir figura

Porém, a NBR 6118, item 15.8.2 diz: Caso seja pilar engastado e livre no topo, o valor de $L_e=2\cdot L$. Nos demais casos, adotar os valores calculados anteriormente pela Equação~\eqref{equacao-comprimento-equivalente}.
