Em lajes de concreto armado armadas em uma direção, a profundidade da linha neutra é encontrada do mesmo jeito da profundidade da linha neutra em vigas de concreto armado, sendo dada pela seguinte equação:

$$x=1,25\cdot d\cdot\left(1-\sqrt{1-\frac{M_d}{0,425\cdot bw\cdot f_{cd}\cdot d^2}} \right )$$

Onde $x$ é a profundidade da linha neutra, $d$ é a altura útil, $M_d$ é o momento fletor de projeto do vão principal, $bw$ é a largura de 1 metro, $f_{cd}$ é a resistência do concreto à compressão de projeto. Aqui é importante tomar cuidado com as unidades.

A NBR 6118/2014 limita a profundidade da linha neutra em $0,45\cdot d$ para concretos com $f_{ck}\leqslant 50$ $MPa$ e em $0,35\cdot d$ para concretos com $50>f_{ck}\leqslant 90$ $MPa$. Essa limitação serve para não deixar que as deformações da laje alcancem o domínio 4 de deformação.