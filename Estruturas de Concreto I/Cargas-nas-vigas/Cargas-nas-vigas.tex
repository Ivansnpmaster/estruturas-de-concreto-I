As cargas verticais nas vigas são:

\begin{itemize}
	\item \textbf{Peso próprio da viga}:

		$$g_{viga}=\gamma_c\cdot bw\cdot h$$

		Onde $g_{viga}$ é o peso próprio da viga em $kN/m^3$, $\gamma_c$ é o peso específico do concreto armado, $bw$ é a largura da seção transversal da viga em $m$ e $h$ é a altura da seção transversal da viga em $m$.

	\item \textbf{Alvenaria sobre a viga}:

		$$g_{alv}=\gamma_{alv}\cdot h\cdot e$$

		Onde $g_{alv}$ é o peso da alvenaria sobre a viga em $kN/m$, $\gamma_{alv}$ é o peso específico da alvenaria que está sobre a viga em $kN/m^3$, $h$ é a altura da parede em $m$ e $e$ é a espessura da parede em $m$.

	\item \textbf{Carga das lajes sobre as vigas}:

		Sendo $P$ a carga nas lajes, devemos distribuir geometricamente $P$ para as vigas. Isso é feito analisando os apoios da laje, pois apoios engastados geralmente recebem mais carga. O primeiro passo é encontrar a área de influência das lajes para as vigas em função dos seus apoios, como na seguinte figura:

	A carga $px$ e $py$, obviamente, serão menores que a carga $P$, entretanto, suas unidades são diferentes. $P$ está em função da área da laje, já $px$ e $py$ estão em função do comprimento da viga.


\end{itemize}