Para o pré-dimensionamento da espessura das lajes maciças, deve-se utilizar a seguinte equação: $$h = \frac{lx}{40}$$ 

Onde $h$ é a altura da laje em $cm$ e $lx$ é menor medida em $cm$ de um dos lados da laje.

As dimensões mínimas especificadas na \textbf{NBR 6118/14} são, em $cm$:

\begin{itemize}
	\item $h \geqslant 7$ para lajes de cobertura (não em balanço);
	\item $h \geqslant 8$ para lajes de piso (não em balanço);
	\item $h \geqslant 10$ para lajes em balanço;
	\item $h \geqslant 10$ para estacionamento para veículos até $30$ $kN$;
	\item $h \geqslant 12$ para estacionamento para veículos com mais de $30$ $kN$.
\end{itemize}

Tentar sempre arredondar para o inteiro superior mais próximo, a fim de facilitar a confecção da forma da laje, sempre se atentando ao mínimo exigido na norma.

Em lajes em balanco, deve-se utilizar um coeficiente de majoração adicional $(\gamma_n)$ na definição do momento fletor de projeto $(M_d)$, esse coeficiente depende da altura da laje em balanço, sendo utilizado para lajes com espessura inferior a 19 $cm$, de acordo com a seguinte tabela:

\begin{table}[H]
	\begin{center}
	\begin{tabular}{l|llllllllll}
	\hline
		$h (cm)$   & 19   & 18   & 17   & 16   & 15   & 14   & 13   & 12   & 11   & 10   \\
		$\gamma_n$ & 1,00 & 1,05 & 1,10 & 1,15 & 1,20 & 1,25 & 1,30 & 1,35 & 1,40 & 1,45 \\ \hline
	\end{tabular}


	\caption{Coeficiente de majoração adicional $(\gamma_n)$ para majoração do momento fletor de projeto.}
	\end{center}
\end{table}